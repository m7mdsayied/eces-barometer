% =========================================================
% PAGE 23: SUB-INDICES PERFORMANCE (Figure 3.1) - ARABIC
% =========================================================

% BLOCK:PAGE_SETUP background=con_bg_ar.png page=23
% 1. Geometry
% Standard margins
\newgeometry{left=2cm, right=2cm, top=3cm, bottom=2.5cm}

% 2. Background Image (con_bg_ar.png)
\begin{tikzpicture}[remember picture, overlay]

    % A. Main Background
    \node[anchor=north west, inner sep=0pt] at (current page.north west) {
        \includegraphics[width=\paperwidth, height=\paperheight]{con_bg_ar.png}
    };

    % B. Page Number (Dynamic) - Left side for Arabic
    \node[anchor=south west] at ([xshift=1cm, yshift=1.2cm]current page.south west) {
        \fontsize{14}{14}\selectfont \textbf{\thepage}
    };

\end{tikzpicture}

% BLOCK:TITLE level=1 color=ecestitle
% ---------------------------------------------------------
% SECTION TITLE
% ---------------------------------------------------------
\noindent
{\fontsize{18}{22}\selectfont \textbf{\color{ecestitle} ثالثاً: تقييم الأداء والتوقعات وفقاً للمؤشرات الفرعية}}

\vspace{1em}

% BLOCK:SUBHEADER_LEGEND text="3.1 تقييم الأداء"
% ---------------------------------------------------------
% SUBSECTION 3.1
% ---------------------------------------------------------
\noindent
{\fontsize{11}{14}\selectfont \textbf{\color{ecestitle} 3.1 تقييم الأداء}}

\vspace{0.8em}

% BLOCK:PARAGRAPH
% ---------------------------------------------------------
% BODY TEXT
% ---------------------------------------------------------
\noindent
{\fontsize{10}{13}\selectfont
ارتفاع مؤشرات الإنتاج والمبيعات المحلية واستغلال الطاقة الإنتاجية عن المستوى المحايد خلال الربع محل الدراسة ولكن كانت أقل من الربع السابق وأن كانت أفضل من الربع المناظر. بينما حققت الشركات الكبيرة قيما أعلى من المستوى المحايد بـ 5 نقاط وأعلى من الربع السابق والمناظر، بينما سجلت في مؤشر المخزون السلعي قيما أقل من المستوى المحايد بـ 7 نقاط وأقل من الربع السابق والمناظر.

\vspace{0.8em}

وحققت الشركات الصغيرة والمتوسطة قيما أعلى من المستوى المحايد بـ 5 نقاط واقل من الربع السابق بـ 5 نقاط ولكن أفضل من الربع المناظر بـ 8 نقاط. بينما سجلت قيما دون المستوى المحايد بـ 8 نقاط في مؤشر المخزون السلعي، قيما أقل من الربع السابق بـ 5 نقاط ولكن أفضل من الربع المناظر بـ 3 نقاط.
}

\vspace{0.5em}

% BLOCK:CHART chart=ch20.png title="الشكل 3-1 (أداء المؤشرات الفرعية)"
% ---------------------------------------------------------
% FIGURE 3.1 (Sub-indices Performance)
% ---------------------------------------------------------
\begin{figure}[h!]
    \centering
    % Contains the Production and Sales Indices chart
    \includegraphics[width=\linewidth]{ch20.png}
\end{figure}

% BLOCK:FOOTNOTE
% ---------------------------------------------------------
% SOURCE & NOTES
% ---------------------------------------------------------
\vspace{-0.5em}
\noindent
{\fontsize{8}{10}\selectfont
المصدر: نتائج الاستبيان.\\
* تم عكس مؤشر المخزون السلعي ليعكس الأثر السلبي الناتج عن ارتفاع المخزون السلعي على الشركات. لذلك فكلما كان المؤشر السلعي مرتفعا كان المخزون السلعي منخفض والعكس صحيح.
}

\clearpage

% =========================================================
% PAGE 24: PRICES, WAGES, INVESTMENT, EMPLOYMENT INDICES - ARABIC
% =========================================================

% BLOCK:PAGE_SETUP background=con_bg_ar.png page=24
% 1. Geometry (Persist)

% 2. Background Image (con_bg_ar.png)
\begin{tikzpicture}[remember picture, overlay]

    % A. Main Background
    \node[anchor=north west, inner sep=0pt] at (current page.north west) {
        \includegraphics[width=\paperwidth, height=\paperheight]{con_bg_ar.png}
    };

    % B. Page Number (Dynamic) - Left side for Arabic
    \node[anchor=south west] at ([xshift=1cm, yshift=1.2cm]current page.south west) {
        \fontsize{14}{14}\selectfont \textbf{\thepage}
    };

\end{tikzpicture}

% BLOCK:PARAGRAPH bold=true
% ---------------------------------------------------------
% SECTION TITLE: CONTINUATION FROM PAGE 23
% ---------------------------------------------------------
\noindent
{\fontsize{10}{13}\selectfont \textbf{تجاوز مؤشر أسعار المنتجات النهائية والأجور لكافة الشركات المستوى المحايد}}

\vspace{0.5em}

% BLOCK:PARAGRAPH
\noindent
{\fontsize{10}{13}\selectfont
تجاوز مؤشر أسعار المنتجات النهائية لكافة الشركات المستوى المحايد ليعكس ثبات الأسعار خلال الربع محل الدراسة. وتجاوز مؤشر الأجور المستوى المحايد بـ 4 نقاط خلال الربع محل الدراسة إلا أنه انخفض مقارنة بالربع السابق (الشكل 3-2).
}

\vspace{1em}

% BLOCK:CHART chart=ch21.png title="الشكل 3-2 (الأسعار/الأجور)"
% ---------------------------------------------------------
% FIGURE 3.2 (Prices/Wages)
% ---------------------------------------------------------
\begin{figure}[h!]
    \centering
    % Price and Production Cost Indices
    \includegraphics[width=\linewidth]{ch21.png}
\end{figure}

% BLOCK:FOOTNOTE
\vspace{-0.5em}
\noindent
{\fontsize{8}{10}\selectfont
المصدر: نتائج الاستبيان.\\
* تم عكس مؤشر أسعار المدخلات ليعكس الأثر السلبي الناتج عن ارتفاع الأسعار على المؤشر الإجمالي. لذلك فكلما كان المؤشر منخفضا كان الارتفاع السعري أكبر والعكس صحيح.
}

\vspace{1.5em}

% BLOCK:PARAGRAPH bold=true
% ---------------------------------------------------------
% SECTION 2: INVESTMENT AND EMPLOYMENT
% ---------------------------------------------------------
\noindent
{\fontsize{11}{13}\selectfont \textbf{استقرار مؤشري الاستثمار والتوظيف لكافة الشركات}}

\vspace{0.6em}

% BLOCK:PARAGRAPH
\noindent
{\fontsize{10}{13}\selectfont
على صعيد الشركات الكبيرة سجل مؤشر الاستثمار خلال الربع محل الدراسة قيما أعلى من المستوى المحايد بنقطة واحدة وأقل من الربع السابق والمناظر. بينما سجل مؤشر التوظيف قيما عند المستوى المحايد وأقل من الربع السابق والمناظر.

\vspace{0.6em}

وبالنسبة للشركات الصغيرة والمتوسطة جاء مؤشر الاستثمار عند المستوى المحايد مسجلا تراجعا عن الربع السابق. إلا أنه سجل نفس قيم الربع المناظر. بينما سجل مؤشر التوظيف قيما أعلى من المستوى المحايد بنقطة واحدة وأعلى من الربع السابق والمناظر (الشكل 3-3).
}

\vspace{0.5em}

% BLOCK:CHART chart=ch22.png title="الشكل 3-3 (الاستثمار/التوظيف)"
% ---------------------------------------------------------
% FIGURE 3.3 (Invest/Employ)
% ---------------------------------------------------------
\begin{figure}[h!]
    \centering
    % Investment and Employment Indices
    \includegraphics[width=\linewidth]{ch22.png}
\end{figure}

% BLOCK:FOOTNOTE
\vspace{-0.5em}
\noindent
{\fontsize{8}{10}\selectfont
المصدر: نتائج الاستبيان.
}

\clearpage

% =========================================================
% PAGE 25: PERFORMANCE EXPECTATIONS (Figure 3.4) - ARABIC
% =========================================================

% BLOCK:PAGE_SETUP background=con_bg_ar.png page=25
% 1. Geometry (Persist)

% 2. Background Image (con_bg_ar.png)
\begin{tikzpicture}[remember picture, overlay]

    % A. Main Background
    \node[anchor=north west, inner sep=0pt] at (current page.north west) {
        \includegraphics[width=\paperwidth, height=\paperheight]{con_bg_ar.png}
    };

    % B. Page Number (Dynamic) - Left side for Arabic
    \node[anchor=south west] at ([xshift=1cm, yshift=1.2cm]current page.south west) {
        \fontsize{14}{14}\selectfont \textbf{\thepage}
    };

\end{tikzpicture}

% BLOCK:SUBHEADER_LEGEND text="3.2 توقعات الأداء"
% ---------------------------------------------------------
% SECTION HEADER
% ---------------------------------------------------------
\noindent
{\fontsize{11}{14}\selectfont \textbf{\color{ecestitle} 3.2 توقعات الأداء}}

\vspace{0.8em}

% BLOCK:PARAGRAPH
% ---------------------------------------------------------
% INTRO TEXT
% ---------------------------------------------------------
\noindent
{\fontsize{10}{13}\selectfont
\textbf{توقعات بارتفاع أداء كافة الشركات عن المستوى المحايد خلال الربع القادم نظرا لتوقعات بثبات معظم المؤشرات الفرعية.}

\vspace{0.8em}

جاءت توقعات الشركات الكبيرة حول مؤشري الإنتاج واستغلال الطاقة الإنتاجية خلال الربع (\NextQuarterText) أعلى من المستوى المحايد ولكن أقل من الربع السابق والمناظر. بينما سجلت التوقعات المتعلقة بالمبيعات المحلية والصادرات قيما أقل من الربع السابق ولكن أعلى من الربع المناظر. وأشارت التوقعات إلى ارتفاع في مؤشر المخزون السلعي عن المستوى المحايد وعن كل من الربع السابق والمناظر.

\vspace{0.8em}

وجاءت توقعات الشركات الصغيرة والمتوسطة للمبيعات المحلية والمخزون أعلى من المستوى المحايد ومقارنة بكل من الربع السابق والمناظر، بينما جاءت توقعات مؤشري الإنتاج واستغلال الطاقة الإنتاجية أقل بنقطة واحدة عن الربع السابق وإن كانت أفضل من الربع المناظر بـ 4 نقاط. بينما كانت توقعات مؤشر الصادرات أقل من الربع السابق والمناظر (الشكل 3-4).
}

\vspace{0.5em}

% BLOCK:CHART chart=ch23.png title="الشكل 3-4 (توقعات حسب الحجم)"
% ---------------------------------------------------------
% FIGURE 3.4 (Expectations Size)
% ---------------------------------------------------------
\begin{figure}[h!]
    \centering
    % Production and Sales Indices by Firm Size - Outlook
    \includegraphics[width=\linewidth]{ch23.png}
\end{figure}

% BLOCK:FOOTNOTE
% ---------------------------------------------------------
% SOURCE & NOTES
% ---------------------------------------------------------
\vspace{-0.5em}
\noindent
{\fontsize{8}{10}\selectfont
المصدر: نتائج الاستبيان.\\
* تم عكس مؤشر المخزون السلعي ليعكس الأثر السلبي الناتج عن ارتفاع المخزون السلعي على الشركات. لذلك فكلما كان المؤشر منخفضا كان المخزون السلعي مرتفعا والعكس صحيح.
}

\vspace{1.5em}

% BLOCK:PARAGRAPH bold=true
% ---------------------------------------------------------
% BOTTOM TEXT
% ---------------------------------------------------------
\noindent
{\fontsize{10}{13}\selectfont \textbf{توقعات بارتفاع أسعار المنتجات النهائية والأجور لكافة الشركات عن المستوى المحايد}}

\vspace{0.8em}

% BLOCK:PARAGRAPH
\noindent
{\fontsize{10}{13}\selectfont
انخفض مؤشر أسعار المنتجات النهائية مقارنة بالربع السابق والمناظر، ولكنه تجاوز المستوى المحايد، مدفوعا بارتفاع مؤشر أسعار المدخلات عن الربع السابق والمناظر.

\vspace{0.8em}

في حين تتوقع الشركات الصغيرة والمتوسطة ثبات مؤشر الأجور عند نفس قيم الربع السابق وأعلى من المستوى المحايد بنقطتين ولكن أقل من الربع المناظر
}

\clearpage

% =========================================================
% PAGE 26: PRICES, WAGES, INVESTMENT OUTLOOK - ARABIC
% =========================================================

% BLOCK:PAGE_SETUP background=con_bg_ar.png page=26
% 1. Geometry (Persist)

% 2. Background Image (con_bg_ar.png)
\begin{tikzpicture}[remember picture, overlay]

    % A. Main Background
    \node[anchor=north west, inner sep=0pt] at (current page.north west) {
        \includegraphics[width=\paperwidth, height=\paperheight]{con_bg_ar.png}
    };

    % B. Page Number (Dynamic) - Left side for Arabic
    \node[anchor=south west] at ([xshift=1cm, yshift=1.2cm]current page.south west) {
        \fontsize{14}{14}\selectfont \textbf{\thepage}
    };

\end{tikzpicture}

% BLOCK:PARAGRAPH
% ---------------------------------------------------------
% TOP TEXT (Continuation)
% ---------------------------------------------------------
\noindent
{\fontsize{10}{13}\selectfont
بنقطة واحدة. وجاءت توقعات الشركات الكبيرة أعلى من الربع السابق بنقطة واحدة ولكنها أقل من الربع المناظر بنقطتين (الشكل 3-5).
}

\vspace{0.5em}

% BLOCK:CHART chart=ch24.png title="الشكل 3-5 (توقعات الأسعار)"
% ---------------------------------------------------------
% FIGURE 3.5 (Expectations Prices)
% ---------------------------------------------------------
\begin{figure}[h!]
    \centering
    % Prices and Production Cost Indices by Firm Size - Outlook
    \includegraphics[width=\linewidth]{ch24.png}
\end{figure}

% BLOCK:FOOTNOTE
% ---------------------------------------------------------
% SOURCE & NOTES FOR FIG 3.5
% ---------------------------------------------------------
\vspace{-0.5em}
\noindent
{\fontsize{8}{10}\selectfont
المصدر: نتائج الاستبيان.\\
* تم عكس مؤشر أسعار المدخلات ليعكس الأثر السلبي الناتج عن ارتفاع الأسعار على المؤشر الإجمالي. لذلك فكلما كان المؤشر منخفضا كان الارتفاع السعري أكبر والعكس صحيح.
}

\vspace{2em}

% BLOCK:PARAGRAPH bold=true
% ---------------------------------------------------------
% MIDDLE SECTION: INVESTMENT AND EMPLOYMENT
% ---------------------------------------------------------
\noindent
{\fontsize{10}{13}\selectfont \textbf{ثبات توقعات مؤشري الاستثمار والتوظيف لكافة الشركات عند المستوى المحايد}}

\vspace{0.8em}

% BLOCK:PARAGRAPH
\noindent
{\fontsize{10}{13}\selectfont
تتوقع كافة الشركات أن يتجاوز مؤشر الاستثمار المستوى المحايد وكذلك الربع السابق والمناظر بنقطة واحدة. كما تتوقع أن يتجاوز مؤشر التوظيف الربع السابق بنقطة واحدة ويماثل الربع المناظر ويسجل قيما أعلى من المستوى المحايد بنقطة واحدة (الشكل 3-6).
}

\vspace{0.5em}

% BLOCK:CHART chart=ch25.png title="الشكل 3-6 (توقعات الاستثمار/التوظيف)"
% ---------------------------------------------------------
% FIGURE 3.6 (Expectations Invest/Employ)
% ---------------------------------------------------------
\begin{figure}[h!]
    \centering
    % Investment and Employment Indices by Firm Size - Outlook
    \includegraphics[width=\linewidth]{ch25.png}
\end{figure}

% BLOCK:FOOTNOTE
% ---------------------------------------------------------
% SOURCE FOR FIG 3.6
% ---------------------------------------------------------
\vspace{-0.5em}
\noindent
{\fontsize{8}{10}\selectfont
المصدر: نتائج الاستبيان.
}

\clearpage
