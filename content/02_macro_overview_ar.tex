% =========================================================
% PAGE 8: MACROECONOMIC OVERVIEW (ARABIC)
% =========================================================

% BLOCK:PAGE_SETUP background=mc_bg_ar.png page=8
% 1. Geometry (Sidebar on Right)
\newgeometry{right=5cm, left=1.5cm, top=3cm, bottom=2.5cm}

% 2. Background Graphics
\begin{tikzpicture}[remember picture, overlay]
    % Background Image (Sidebar on Right)
    \node[anchor=north west, inner sep=0pt] at (current page.north west) {
        \includegraphics[width=\paperwidth, height=\paperheight]{mc_bg_ar.png}
    };

    % Page Number (Bottom Left for RTL)
    \node[anchor=south west] at ([xshift=1cm, yshift=1.2cm]current page.south west) {
        \fontsize{14}{14}\selectfont \textbf{\thepage}
    };
\end{tikzpicture}

% BLOCK:TITLE level=1 color=ecestitle
% ---------------------------------------------------------
% TITLE & INTRO
% ---------------------------------------------------------
\noindent
{\fontsize{24}{28}\selectfont \textbf{\color{ecestitle}
نظرة عامة على الاقتصاد الكلي
}}

\vspace{1em}

% BLOCK:PARAGRAPH bold=true
\noindent
{\fontsize{11}{14}\selectfont \textbf{
    عالمياً: استمرار توقعات تراجع النمو العالمي، نمو منخفض ومتفاوت وتحديات متعددة تواجه الاقتصاد العالمي
    }}

\vspace{0.8em}

% BLOCK:PARAGRAPH
% ---------------------------------------------------------
% BODY TEXT
% ---------------------------------------------------------
\noindent
{\fontsize{10}{13}\selectfont
استمر تراجع توقعات النمو العالمي ليتراوح ما بين 2.3 - 3\% خلال عام 2025 على أثر استمرار حالة عدم اليقين بسبب السياسات التجارية الحمائية والتوترات الجيوسياسية. ويعكس هذا التراجع تباطؤ معدلات النمو المتوقعة لتبلغ 1.5\% للاقتصادات المتقدمة وتستقر عند 4\% للأسواق الصاعدة والنامية.\textsuperscript{1} علاوة على توقعات بتراجع حاد في معدلات التجارة الدولية وتدفقات الاستثمار وزيادة أعباء الاقتراض.\textsuperscript{2}

\vspace{0.8em}

توقع استمرار المسار النزولي للتضخم العالمي من متوسط سنوي قدره 5.9\% في 2024 إلى 4.2\% في 2025 نتيجة انخفاض أسعار السلع الأساسية،\textsuperscript{3} والانخفاض النسبي في أسعار الطاقة بعد زيادة إنتاج النفط من خارج تحالف الأوبك+ وضعف الطلب العالمي. ومع ذلك، لا تزال المخاطر تحيط بمسار التضخم والنمو خاصة حال استمرار التوترات الجيوسياسية وتصاعد الحمائية التجارية وصدمات تغير المناخ.\textsuperscript{4} وعليه، اعتمدت البنوك المركزية في الاقتصادات المتقدمة والناشئة على حد سواء نهجا حذرا بشأن السياسة النقدية.\textsuperscript{5}

\vspace{0.8em}

واصل المؤشر المركب لمديري المشتريات النمو للشهر السادس على التوالي مسجلا (51.7 نقطة) في يونيو 2025 مدفوعا بنمو إنتاج الصناعات التحويلية، ونمو بطيء لنشاط الخدمات. وبذلك تكون قيمة المؤشر خلال الربع الثاني من عام 2025 في حدود 51.2 نقطة وهو ما يمثل تراجعا بنحو 0.6 عن الربع السابق وبنحو 1.8 عن نفس الربع في عام 2024. وبالرغم من ارتفاع المؤشر عن المستوى المحايد إلا أن مستويات الثقة لدى قطاعي الصناعات التحويلية والخدمات شهدت انخفاضا كما شهد الطلب على العمالة تراجعا ملحوظا ليعكس الضغوط الاقتصادية العالمية.\textsuperscript{6}

\vspace{0.8em}

\textbf{محلياً: مؤشرات لاستقرار اقتصادي كلي خلال النصف الأول من عام 2025 بسبب تحسن السياحة، واستثمارات رأس الحكمة، وتقلص فجوة الصرف الأجنبي. استكمال الإصلاحات الهيكلية ضرورة لتعظيم الاستفادة من الاستقرار الاقتصادي الكلي وتعزيز صمود الاقتصاد المصري في مواجهة الأزمات.}

\vspace{0.8em}

انتهاء المراجعة الرابعة لصندوق النقد الدولي والتي بموجبها سيتم صرف دفعة بقيمة 1.2 مليار دولار لاستكمال برنامج الإصلاح الاقتصادي. ويتوقع الصندوق استمرار النمو الاقتصادي ليكون في حدود 3.6\% بنهاية 2025/2024 على أثر تحسن الاستثمارات الأجنبية المباشرة في الساحل الشمالي، وفي قطاع التعدين والصناعة التحويلية، وكذلك توقعات بتراجع معدل التضخم وتحسن صافي الاحتياطيات من النقد الأجنبي، وتراجع نسبة الدين إلى الناتج على أثر استثمارات رأس الحكمة، إلا أنه لا يزال عبء الدين مرتفعا مما يضيق من الحيز المالي للإنفاق الاجتماعي المستهدف.

\vspace{0.8em}

للبناء على الاستقرار الاقتصادي وتعزيز الصمود في مواجهة التحديات العالمية، يؤكد الصندوق على ضرورة استكمال الإصلاحات الهيكلية، ومن أهمها: سرعة تخارج الدولة من الأنشطة غير الاستراتيجية، واستكمال رفع دعم الطاقة، وحوكمة ورقابة الهيئات خارج الموازنة، واستمرار تهيئة بيئة الأعمال بما يضمن عدالة المنافسة بين الدولة والقطاع الخاص. وأخيرا، دمج جهود مواجهة التغيرات المناخية في الخطط الاستراتيجية.
}

% BLOCK:FOOTNOTE
% ---------------------------------------------------------
% FOOTNOTES SECTION
% ---------------------------------------------------------
\vfill

\noindent
\rule{5cm}{0.4pt}

\vspace{0.3em}

\noindent
{\fontsize{7}{9}\selectfont
\textsuperscript{1} IMF, World economic outlook, July 2025\\
\textsuperscript{2} World Bank, Global Economic Prospect, June 2025\\
\textsuperscript{3} نفس المرجع.\\
\textsuperscript{4} IMF, World economic outlook, July 2025\\
\textsuperscript{5} نفس المرجع.\\
\textsuperscript{6} \underline{\href{https://www.spglobal.com/marketintelligence/en/mi/research-analysis/global-economic-expansion-gains-momentum-in-march-Apr24.html}{\color{eceslink} https://www.spglobal.com/marketintelligence/en/mi/research-analysis/global-economic-expansion-gains-momentum-in-march-Apr24.html}}
}

\clearpage

% =========================================================
% PAGE 9: MACROECONOMIC INDICATORS (ARABIC)
% =========================================================

% BLOCK:PAGE_SETUP background=mc_bg_ar.png page=9
% 1. Geometry (Sidebar on Right)
\newgeometry{right=5cm, left=1.5cm, top=3cm, bottom=2.5cm}

% 2. Background Graphics
\begin{tikzpicture}[remember picture, overlay]
    \node[anchor=north west, inner sep=0pt] at (current page.north west) {
        \includegraphics[width=\paperwidth, height=\paperheight]{mc_bg_ar.png}
    };
    \node[anchor=south west] at ([xshift=1cm, yshift=1cm]current page.south west) {
        \fontsize{14}{14}\selectfont \textbf{\thepage}
    };
\end{tikzpicture}

% BLOCK:PARAGRAPH
% ---------------------------------------------------------
% INTRO TEXT
% ---------------------------------------------------------
\noindent
{\fontsize{10}{13}\selectfont
ويتضمن الجزء التالي عرضا موجزا لتطورات أهم المؤشرات الاقتصادية الكلية وفقا لأحدث البيانات المنشورة حتى تاريخ نشر هذا التقرير.
}

\vspace{0.8em}

% BLOCK:SUBHEADER_LEGEND text="أولا: معدل نمو الناتج المحلي الإجمالي"
% ---------------------------------------------------------
% SECTION I: GDP GROWTH
% ---------------------------------------------------------
\noindent
{\fontsize{11}{14}\selectfont \textbf{أولا: معدل نمو الناتج المحلي الإجمالي}}

% BLOCK:PARAGRAPH
\noindent
{\fontsize{10}{13}\selectfont
سجل الناتج المحلي الإجمالي نموا بمعدل 4.7\% في الربع الثالث من العام المالي 2025/2024 (يناير - مارس 2025) بما يمثل ارتفاعا بنحو 11\% مقارنة بالربع السابق له و115\% مقارنة بالربع المناظر من العام المالي السابق 2024/2023. ويعود هذا الارتفاع إلى المساهمة الموجبة لصافي الصادرات في النمو والتي عوضت التراجع الطفيف في مساهمة الاستهلاك العائلي، بالإضافة إلى التعافي في نمو قطاعات الصناعات التحويلية غير البترولية، والسياحة، والاتصالات وتكنولوجيا المعلومات. وقد حد من هذا النمو استمرار تراجع نشاط قناة السويس على خلفية استمرار التوترات الجيوسياسية، وكذلك تراجع أداء قطاع الاستخراجات.\textsuperscript{7}
}

% BLOCK:CHART chart=ch7.png title="الشكل 1: النمو في الناتج المحلي الإجمالي"
% ---------------------------------------------------------
% FIGURE 1: ch7.png
% ---------------------------------------------------------
\vspace{0.5em}
\noindent
{\fontsize{10}{12}\selectfont \textbf{\color{ecestitle} الشكل 1: النمو في الناتج المحلي الإجمالي الحقيقي خلال الفترة (2016/2015 -- الربع الثالث 2025/2024)}}

\noindent
\includegraphics[width=\linewidth]{ch7.png}

\noindent
{\fontsize{8}{10}\selectfont \textit{المصدر: وزارة التخطيط والتنمية الاقتصادية والتعاون الدولي.}}

\vspace{1.5em}

% BLOCK:SUBHEADER_LEGEND text="ثانيا: معدل التضخم"
% ---------------------------------------------------------
% SECTION II: INFLATION
% ---------------------------------------------------------
\noindent
{\fontsize{11}{14}\selectfont \textbf{ثانيا: معدل التضخم}}

% BLOCK:PARAGRAPH
\noindent
{\fontsize{10}{13}\selectfont
بعد ارتفاع متوسط المعدل السنوي للتضخم العام لإجمالي الجمهورية على مدى الثلاثة أشهر السابقة، تراجع ليبلغ 14.4\% خلال شهر يونيو مقارنة بنحو 16.5\% لشهر مايو 2025. ويعكس ذلك انحسار الصدمات التي نتجت عن رفع أسعار الوقود في أبريل 2025.\textsuperscript{8} ومن ثم ارتفاع تكاليف النقل البري بالإضافة إلى ارتفاع تكاليف الخدمات والسلع الاستهلاكية، خاصة المنتجات الصيدلانية، وتلاشي أثر الارتفاع الموسمي للطلب على السلع الغذائية بعد انتهاء شهر رمضان والأعياد (الجهاز المركزي للتعبئة العامة والإحصاء، 2025).

\vspace{0.8em}

ولاستدامة المسار النزولي للتضخم قررت لجنة السياسة النقدية للبنك المركزي المصري في اجتماعها يوم الخميس الموافق 10 يوليو 2025 الإبقاء على سعري عائد الإيداع والإقراض لليلة واحدة وسعر العملية الرئيسية للبنك المركزي عند 24.00\% و25.00\% و24.50\%، على الترتيب. كما قررت الإبقاء على سعر الائتمان والخصم عند 24.50\%.\textsuperscript{9}
}

% BLOCK:FOOTNOTE
% ---------------------------------------------------------
% FOOTNOTES
% ---------------------------------------------------------
\vfill

\noindent
\rule{5cm}{0.4pt}

\vspace{0.3em}

\noindent
{\fontsize{7}{9}\selectfont
\textsuperscript{7} وزارة التخطيط والتنمية الاقتصادية والتعاون الدولي، 2025. النشرة ربع السنوية للناتج المحلي الإجمالي الربع الثالث من العام المالي 2025/2024، تحليل معدلات النمو والاستثمار والأداء القطاعي.\\
\textsuperscript{8} الجهاز المركزي للتعبئة العامة والإحصاء النشرة الشهرية للأرقام القياسية لأسعار المستهلكين، أعداد مختلفة.\\
\textsuperscript{9} البنك المركزي المصري 2025 بيان صحفي لجنة السياسة النقدية في 10 يوليو 2025.
}

\clearpage

% =========================================================
% PAGE 10: INFLATION CHART & FOREIGN TRANSACTIONS
% =========================================================

% BLOCK:PAGE_SETUP background=mc_bg_ar.png page=10
% 1. Geometry (Sidebar on Right)
\newgeometry{right=5cm, left=1.5cm, top=3cm, bottom=2.5cm}

% 2. Background Graphics
\begin{tikzpicture}[remember picture, overlay]
    \node[anchor=north west, inner sep=0pt] at (current page.north west) {
        \includegraphics[width=\paperwidth, height=\paperheight]{mc_bg_ar.png}
    };
    \node[anchor=south west] at ([xshift=1cm, yshift=1cm]current page.south west) {
        \fontsize{14}{14}\selectfont \textbf{\thepage}
    };
\end{tikzpicture}

% BLOCK:CHART chart=ch8.png title="الشكل 2: التضخم وأسعار الفائدة"
% ---------------------------------------------------------
% FIGURE 2: ch8.png
% ---------------------------------------------------------
\noindent
{\fontsize{10}{12}\selectfont \textbf{\color{ecestitle} الشكل 2: التضخم وأسعار الفائدة الأساسية}}

\noindent
\includegraphics[width=\linewidth]{ch8.png}

\noindent
{\fontsize{7}{9}\selectfont \textit{المصادر: البنك المركزي المصري، النشرة الإحصائية الشهرية، و بيان صحفي للجنة السياسة النقدية في 22 مايو 2025؛ والجهاز المركزي للتعبئة العامة والإحصاء، النشرة الشهرية لأسعار المستهلكين، أعداد مختلفة.}}

\vspace{1.5em}

% BLOCK:SUBHEADER_LEGEND text="ثالثا: المعاملات الخارجية"
% ---------------------------------------------------------
% SECTION III: FOREIGN TRANSACTIONS
% ---------------------------------------------------------
\noindent
{\fontsize{11}{14}\selectfont \textbf{ثالثا: المعاملات الخارجية}}

\vspace{0.5em}

% BLOCK:PARAGRAPH
\noindent
{\fontsize{10}{13}\selectfont
\textbf{ميزان المدفوعات:} شهدت الفترة يوليو/مارس من السنة المالية 2025/2024 تحول ميزان المدفوعات من فائض كلي بلغ نحو 4.1 مليار دولار إلى عجز كلي قدره 1.9 مليار دولار. ويرجع ذلك إلى تراجع صافي التدفقات للداخل في المعاملات الرأسمالية والمالية إلى 7.7 مليار دولار خلال الفترة المذكورة مقابل 20 مليار دولار خلال الفترة يوليو - مارس 2024/2023 بسبب استثمارات رأس الحكمة.

\vspace{0.6em}

تراجع عجز حساب المعاملات الجارية بنسبة 22.6\% ليبلغ نحو 13.2 مليار دولار خلال الفترة يوليو-مارس 2025/2024 مقابل 17.1 مليار في نفس الفترة من العام المالي السابق بسبب زيادة تحويلات المصريين العاملين بالخارج بنسبة 82.7\% لتبلغ 26.4 مليار دولار، ونمو الإيرادات السياحية بنسبة 23.0\% لتسجل 12.5 مليار دولار. وساهمت الزيادة الكبيرة في الصادرات السلعية غير البترولية بنسبة 56.9\% لتسجل 25.6 مليار دولار.

\vspace{0.6em}

وقد حد من تحسن حساب المعاملات الجارية ارتفاع العجز في الميزان التجاري البترولي لارتفاع الواردات البترولية إلى نحو 10.3 مليار دولار، وكذلك الواردات غير البترولية لتبلغ 28 مليار دولار، وانخفاض إيرادات قناة السويس بمعدل 54.1\% لتقتصر على 2.6 مليار دولار (البنك المركزي المصري، 2025).\textsuperscript{10}

\vspace{0.6em}

\textbf{صافي الأصول الأجنبية:} ارتفع صافي الأصول الأجنبية لدى القطاع المصرفي (البنك المركزي والبنوك التجارية) بنحو 1.2 مليار دولار ليصل إلى 14.7 مليار دولار في مايو 2025، مقارنة بـ 13.6 مليار دولار في أبريل 2025. وجاء هذا الارتفاع مدفوعا بالتحسن في صافي الأصول الأجنبية لدى البنوك التجارية ليبلغ أعلى مستوى له منذ فبراير 2021 (رئاسة مجلس الوزراء، 2025).\textsuperscript{11}

\vspace{0.6em}

\textbf{صافي الاحتياطيات الدولية:} ارتفع صافي الاحتياطيات الدولية إلى 48.700 مليار دولار في نهاية يونيو 2025 مقابل 48.525 مليار دولار بنهاية شهر مايو وبذلك ارتفع متوسط صافي الاحتياطي خلال الربع الأخير من العام المالي 2025/2024 (\CurrentQuarterText) بنحو 2\% مقارنة بالربع السابق، وبنحو 9\% عن الربع المناظر من العام المالي 2024/2023.

\vspace{0.6em}

استقرت قيمة الجنيه أمام الدولار خلال الربع محل الدراسة (\CurrentQuarterText) عند 50.38. ويعد ذلك تعافي بنحو 0.1\% للجنيه مقارنة بالربع السابق، بينما يعد تراجعا بنحو 6\% عن قيمته في الربع المناظر من العام المالي 2024/2023 (الشكل 3).
}

% BLOCK:FOOTNOTE
% ---------------------------------------------------------
% FOOTNOTES
% ---------------------------------------------------------
\vfill

\noindent
\rule{5cm}{0.4pt}

\vspace{0.3em}

\noindent
{\fontsize{7}{9}\selectfont
\textsuperscript{10} البنك المركزي المصري، 2025. البيان الصحفي بشأن أداء ميزان المدفوعات خلال الفترة يوليو-مارس من السنة المالية 2025/2024.\\
\textsuperscript{11} رئاسة مجلس الوزراء، 2025. ارتفاع صافي الأصول الأجنبية لدى القطاع المصرفي بنحو 1.2 مليار دولار ليصل إلى 14.7 مليار دولار في مايو 2025 مقارنة بمستواه البالغ 13.6 مليار دولار في الشهر السابق.
}

\clearpage

% =========================================================
% PAGE 11: PUBLIC FINANCE (ARABIC)
% =========================================================

% BLOCK:PAGE_SETUP background=mc_bg_ar.png page=11
% 1. Geometry (Sidebar on Right)
\newgeometry{right=5cm, left=1.5cm, top=3cm, bottom=2.5cm}

% 2. Background Graphics
\begin{tikzpicture}[remember picture, overlay]
    \node[anchor=north west, inner sep=0pt] at (current page.north west) {
        \includegraphics[width=\paperwidth, height=\paperheight]{mc_bg_ar.png}
    };
    \node[anchor=south west] at ([xshift=1cm, yshift=1cm]current page.south west) {
        \fontsize{14}{14}\selectfont \textbf{\thepage}
    };
\end{tikzpicture}

% BLOCK:CHART chart=ch9.png title="الشكل 3: صافي الاحتياطيات الدولية"
% ---------------------------------------------------------
% FIGURE 3: ch9.png
% ---------------------------------------------------------
\noindent
{\fontsize{10}{12}\selectfont \textbf{\color{ecestitle} الشكل 3: صافي الاحتياطيات الدولية وسعر الصرف}}

\noindent
\includegraphics[width=\linewidth]{ch9.png}

\noindent
{\fontsize{7}{9}\selectfont \textit{المصادر: البنك المركزي المصري، النشرة الإحصائية الشهرية، أعداد مختلفة؛ ووزارة المالية، التقرير المالي الشهري، أعداد مختلفة.}}

\vspace{2em}

% BLOCK:SUBHEADER_LEGEND text="رابعا: المالية العامة"
% ---------------------------------------------------------
% SECTION IV: PUBLIC FINANCE
% ---------------------------------------------------------
\noindent
{\fontsize{11}{14}\selectfont \textbf{رابعا: المالية العامة}}

\vspace{0.5em}

% BLOCK:PARAGRAPH
\noindent
{\fontsize{10}{13}\selectfont
انخفض العجز الكلي في الموازنة العامة للدولة كنسبة من الناتج المحلي الإجمالي خلال الفترة يوليو - مايو 2025/2024 إلى -6.9\% مقابل -7.3\% في الفترة المناظرة من العام المالي السابق ويرجع ذلك إلى:
}

% BLOCK:BULLET_LIST
\begin{itemize} \itemsep0.6em
    \item \fontsize{10}{13}\selectfont ارتفاع الإيرادات العامة بنسبة 1.3\% لتسجل 2246.6 مليار جنيه بسبب نمو الإيرادات الضريبية بنسبة 36\%.

    \item \fontsize{10}{13}\selectfont زيادة المصروفات العامة بنحو 25.2\% لتسجل 3408.7 مليار جنيه واستمرار ضبط الإنفاق العام بسقف تريليون جنيه (وزارة المالية 2025).\textsuperscript{12}

    \item \fontsize{10}{13}\selectfont وفقا لوزارة المالية، انخفضت نسبة الدين إلى الناتج المحلي الإجمالي من 89.4\% في يونيو 2024 إلى 85\% في يونيو 2025؛ نتيجة تراجع الدين المحلي بنحو ثمان نقاط مئوية، والدين الخارجي بنحو 2.7 نقطة.
\end{itemize}

% BLOCK:FOOTNOTE
% ---------------------------------------------------------
% FOOTNOTES
% ---------------------------------------------------------
\vfill

\noindent
\rule{5cm}{0.4pt}

\vspace{0.3em}

\noindent
{\fontsize{7}{9}\selectfont
\textsuperscript{12} وزارة المالية، التقرير المالي الشهري، يوليو 2025.
}

\clearpage
