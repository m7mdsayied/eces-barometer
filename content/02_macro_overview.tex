% =========================================================
% PAGE 8: MACROECONOMIC OVERVIEW
% =========================================================

% BLOCK:PAGE_SETUP background=mc_bg.png page=8
% 1. Geometry
% Use the Sidebar Layout margins (Left margin clears the graphic)
\newgeometry{left=5cm, right=1.5cm, top=3.5cm, bottom=2.5cm}

% 2. Background Graphics
\begin{tikzpicture}[remember picture, overlay]
    % Background Image (Contains the person graphic, sidebar, and footer)
    \node[anchor=north west, inner sep=0pt] at (current page.north west) {
        \includegraphics[width=\paperwidth, height=\paperheight]{mc_bg.png}
    };

    % Page Number (Dynamic)
    \node[anchor=south east] at ([xshift=-1cm, yshift=1cm]current page.south east) {
        \fontsize{14}{14}\selectfont \textbf{\thepage}
    };
\end{tikzpicture}

% BLOCK:TITLE level=1 color=ecestitle
% ---------------------------------------------------------
% TITLE & INTRO
% ---------------------------------------------------------
\noindent
{\fontsize{21}{25}\selectfont \textbf{\color{ecestitle}
Macroeconomic Overview
}}

\vspace{0.8em}

% BLOCK:PARAGRAPH bold=true
\noindent
{\fontsize{10}{13}\selectfont \textbf{
    Globally: Global growth forecasts continue their downward trend, with low and uneven growth and multiple challenges facing the global economy.
    }}

\vspace{0.8em}

% BLOCK:PARAGRAPH
% ---------------------------------------------------------
% BODY TEXT
% ---------------------------------------------------------
\noindent
{\fontsize{9}{11}\selectfont
Global growth forecasts continued to decline, ranging between 2.3\% and 3\% in 2025, due to persistent uncertainty resulting from protectionist trade policies and geopolitical tensions. This decline reflects the slowdown in expected growth rates, which are set to reach 1.5\% for advanced economies and stabilize at 4\% for emerging and developing markets.\textsuperscript{1} This is compounded by expectations of a sharp decline in international trade volumes and investment flows, as well as increased borrowing burdens.\textsuperscript{2}

\vspace{0.8em}

Global inflation is expected to continue its downward trajectory, from an annual average of 5.9\% in 2024 to 4.2\% in 2025, as a result of lower commodity prices,\textsuperscript{3} and relative decline of energy prices following increased oil production outside the OPEC+ alliance and weak global demand. However, risks still surround the paths of inflation and growth, especially if geopolitical tensions, rising trade protectionism, and climate change shocks persist.\textsuperscript{4} Accordingly, central banks in both advanced and emerging economies have adopted a cautious approach to monetary policy.\textsuperscript{5}

\vspace{0.8em}

The composite Purchasing Managers' Index (PMI) continued to grow for the sixth month in a row, recording 51.7 points in June 2025, driven by growth in manufacturing production and slower growth in services activity. Thus, the index for the second quarter of 2025 is expected to be approximately 51.2 points, representing a decline of about 0.6 points compared to the previous quarter and around 1.8 points compared to the corresponding quarter in 2024. Although the index surpassed the neutral level, confidence in the manufacturing and services sectors declined, and labor demand saw a significant decline, reflecting global economic pressures.\textsuperscript{6}

\vspace{0.8em}

\textbf{Locally:}
There are signs of macroeconomic stability during the first half of 2025 due to improved tourism, Ras El-Hekma investments, and a narrowing foreign exchange gap. However, completing structural reforms is essential to maximize benefits from macroeconomic stability and enhance the resilience of the Egyptian economy in the face of crises.

\vspace{0.8em}

The fourth review by the International Monetary Fund (IMF) has been concluded, under which a \$1.2 billion tranche would be disbursed to complete the economic reform program. The Fund expects economic growth to continue at around 3.6\% by end of 2024/25, driven by improved foreign direct investment in the North Coast and the mining and manufacturing sectors. It also expects inflation to decline, net foreign exchange reserves to improve, and debt-to-GDP ratio to decline, driven by investments in Ras El-Hekma. However, the debt burden remains high, narrowing the fiscal space for targeted social spending.

\vspace{0.8em}

To build on economic stability and enhance resilience in the face of global challenges, the Fund stressed the need to complete structural reforms. Most importantly, the state must rapidly withdraw from non-strategic activities, complete the removal of energy subsidies, improve governance and oversight of off-budget entities, and continue to avail a business environment that ensures fair competition between the state and the private sector. Finally, there is a need to integrate climate change efforts into strategic plans.
}

% BLOCK:FOOTNOTE
% ---------------------------------------------------------
% FOOTNOTES SECTION
% ---------------------------------------------------------
\vfill % Push to bottom

\noindent
\rule{5cm}{0.4pt} % Horizontal line

\vspace{0.3em}

\noindent
{\fontsize{7}{9}\selectfont
\textsuperscript{1}
IMF, World Economic Outlook July 2025.\\
\textsuperscript{2}
World Bank, Global Economic Prospects June 2025.\\
\textsuperscript{3}
Ibid.\\
\textsuperscript{4}
IMF, World Economic Outlook July 2025\\
\textsuperscript{5}
Ibid.\\
\textsuperscript{6} \underline{\href{https://www.spglobal.com/marketintelligence/en/mi/research-analysis/global-economic-expansion-gains-momentum-in-march-Apr24.html}{\color{eceslink} https://www.spglobal.com/marketintelligence/en/mi/research-analysis/global-economic-expansion-gains-momentum-in-march-Apr24.html}}
}

\clearpage

% =========================================================
% PAGE 9: MACROECONOMIC INDICATORS (GDP & INFLATION)
% =========================================================

% BLOCK:PAGE_SETUP background=mc_bg.png page=9
% 1. Geometry (Maintains sidebar layout)
\newgeometry{left=5cm, right=1.5cm, top=3.5cm, bottom=2.5cm}

% 2. Background Graphics
\begin{tikzpicture}[remember picture, overlay]
    % Background Image
    \node[anchor=north west, inner sep=0pt] at (current page.north west) {
        \includegraphics[width=\paperwidth, height=\paperheight]{mc_bg.png}
    };

    % Page Number (Dynamic)
    \node[anchor=south east] at ([xshift=-1cm, yshift=1cm]current page.south east) {
        \fontsize{14}{14}\selectfont \textbf{\thepage}
    };
\end{tikzpicture}

% BLOCK:PARAGRAPH
% ---------------------------------------------------------
% INTRO TEXT
% ---------------------------------------------------------
\noindent
{\fontsize{10}{13}\selectfont
The following section includes a brief overview of developments in the most important macroeconomic indicators, based on the latest data available up to the date of this report.
}

\vspace{0.5em}

% BLOCK:SUBHEADER_LEGEND text="I. GDP Growth"
% ---------------------------------------------------------
% SECTION I: GDP GROWTH
% ---------------------------------------------------------
\noindent
{\fontsize{11}{14}\selectfont \textbf{I. GDP Growth}}

% BLOCK:BULLET_LIST
\begin{itemize}
    \item \fontsize{10}{13}\selectfont
    GDP grew at a rate of 4.7\% in the third quarter of fiscal year \CurrentFiscalYear{} (January - March 2025), representing an increase of approximately 11\% compared to the previous quarter and 115\% compared to the corresponding quarter of the previous fiscal year (2023/2024). This increase is due to the positive contribution of net exports to growth, which offset the slight decline in the contribution of household consumption, in addition to the recovery of growth in non-oil manufacturing, tourism, and communications and information technology. Growth was inhibited by the continued decline in Suez Canal activity due to ongoing geopolitical tensions, as well as the declining performance of the extraction sector.\textsuperscript{10}
\end{itemize}

% BLOCK:CHART chart=ch6.png title="Figure 1: Real GDP Growth"
% ---------------------------------------------------------
% FIGURE 1
% ---------------------------------------------------------
\vspace{0.2em}
\noindent
{\fontsize{10}{12}\selectfont \textbf{\color{ecestitle} Figure 1: Real GDP Growth (2015/2016 -- Q2 \CurrentFiscalYear)}}

\vspace{0.2em}

\noindent
\includegraphics[width=\linewidth]{ch6.png}

\noindent
{\fontsize{8}{10}\selectfont \textit{Source: Ministry of Planning and Economic Development}}

\vspace{1em}

% BLOCK:SUBHEADER_LEGEND text="II. Inflation"
% ---------------------------------------------------------
% SECTION II: INFLATION
% ---------------------------------------------------------
\noindent
{\fontsize{11}{14}\selectfont \textbf{II. Inflation}}

% BLOCK:BULLET_LIST
\begin{itemize}
    \item \fontsize{10}{13}\selectfont
    Following the increased average annual headline inflation rate for the entire republic over the previous three months, it declined to 14.4\% in June, compared to about 16.5\% in May 2025. This reflects the fading impact of the shocks resulting from the fuel price hike in April 2025,\textsuperscript{12} which increased the cost of land transportation, services and consumer goods, especially pharmaceutical products. It also mirrors the fading effect of the seasonal increase in demand for food commodities after the end of Ramadan and the holidays (Central Agency for Public Mobilization and Statistics, 2025).
\end{itemize}

\vspace{0.5em}

% BLOCK:PARAGRAPH
\noindent
{\fontsize{10}{13}\selectfont
To sustain the downward trend in inflation, the Monetary Policy Committee of the Central Bank of Egypt decided, in its meeting on Thursday, July 10, 2025, to maintain the overnight deposit and lending rates, and the rate of the Central Bank's main operation, at 24.00\%, 25.00\%, and 24.50\%, respectively. It also decided to keep the credit and discount rate at 24.50\%. \textsuperscript{13}
}

% BLOCK:FOOTNOTE
% ---------------------------------------------------------
% FOOTNOTES
% ---------------------------------------------------------
\vfill

\noindent
\rule{5cm}{0.4pt}

\vspace{0.3em}

\noindent
{\fontsize{7}{9}\selectfont
\textsuperscript{10} Ministry of Planning, Economic Development and International Cooperation, 2025 Quarterly GDP Bulletin for the third quarter of fiscal year \CurrentFiscalYear: Analysis of growth rates, investment and sectoral performance.\\
\textsuperscript{12} Central Agency for Public Mobilization and Statistics, Monthly Bulletin of Consumer Price Indices, various issues.\\
\textsuperscript{13} Central Bank of Egypt, 2025, Monetary Policy Committee Press Release, July 10, 2025.
}

\clearpage

% =========================================================
% PAGE 10: INFLATION CHART & FOREIGN TRANSACTIONS
% =========================================================

% BLOCK:PAGE_SETUP background=mc_bg.png page=10
% 1. Geometry (Maintains sidebar layout)
\newgeometry{left=5cm, right=1.5cm, top=3cm, bottom=2.5cm}

% 2. Background Graphics
\begin{tikzpicture}[remember picture, overlay]
    % Background Image
    \node[anchor=north west, inner sep=0pt] at (current page.north west) {
        \includegraphics[width=\paperwidth, height=\paperheight]{mc_bg.png}
    };

    % Page Number (Dynamic)
    \node[anchor=south east] at ([xshift=-1cm, yshift=1cm]current page.south east) {
        \fontsize{14}{14}\selectfont \textbf{\thepage}
    };
\end{tikzpicture}

% BLOCK:CHART chart=ch7.png title="Figure 2: Inflation and Key Interest Rates"
% ---------------------------------------------------------
% FIGURE 2: INFLATION AND RATES
% ---------------------------------------------------------
\noindent
{\fontsize{10}{12}\selectfont \textbf{\color{ecestitle} Figure 2: Inflation and Key Interest Rates}}

\vspace{0.2em}

\noindent
\includegraphics[width=\linewidth]{ch7.png}

\noindent
{\fontsize{7}{9}\selectfont \textit{Sources: Central Bank of Egypt, Monthly Statistical Bulletin; Press release on the exceptional meeting of the Monetary Policy Committee on March 6, 2024; CAPMAS, Monthly Bulletin of Consumer Price Indices, various issues.}}

\vspace{1em}

% BLOCK:SUBHEADER_LEGEND text="III. Foreign transactions"
% ---------------------------------------------------------
% SECTION III: FOREIGN TRANSACTIONS
% ---------------------------------------------------------
\noindent
{\fontsize{11}{14}\selectfont \textbf{III. Foreign transactions}}

\vspace{0.5em}

% BLOCK:PARAGRAPH
\noindent
{\fontsize{10}{13}\selectfont
\textbf{Balance of payments:} The July-March period of fiscal year \CurrentFiscalYear{} saw the balance of payments shift from an overall surplus of approximately \$4.1 billion to an overall deficit of \$1.9 billion. This is due to a decline in net capital inflows and financial transactions to \$7.7 billion during the mentioned period, compared to \$20 billion during the period July-March 2023/2024, thanks to investments in Ras-El-Hekma.

\vspace{0.6em}

The current account deficit declined by 22.6\% to approximately \$13.2 billion during July-March \CurrentFiscalYear, compared to \$17.1 billion during the same period of the previous fiscal year, due to an 82.7\% increase in remittances from Egyptians working abroad, reaching \$26.4 billion, and growth by 23.0\% in tourism revenues, recording \$12.5 billion. The significant increase in non-oil commodity exports contributed to a 56.9\% increase, reaching \$25.6 billion.

\vspace{0.6em}

The improvement in the current account was offset by the increased deficit in the petroleum trade balance due to the rise in petroleum imports to about \$10.3 billion, and non-petroleum imports to \$28 billion, as well as the decline in Suez Canal revenues by 54.1\% to reach \$2.6 billion only (Central Bank of Egypt, 2025).\textsuperscript{17}

\vspace{0.6em}

\textbf{Net foreign assets:} Net foreign assets of the banking sector (both central bank and commercial banks) rose by approximately \$1.2 billion to \$14.7 billion in May 2025, compared to \$13.6 billion in April 2025. This increase was driven by an improvement in net foreign assets at commercial banks, reaching their highest level since February 2021 (Cabinet, 2025).\textsuperscript{18}

\vspace{0.6em}

\textbf{Net international reserves:} Net international reserves grew to \$48.7 billion at the end of June 2025, compared to \$48.525 billion at the end of May. Thus, average net reserves during the last quarter of fiscal year \CurrentFiscalYear{} (\CurrentQuarterText) increased by approximately 2\% compared to the previous quarter, and by approximately 9\% compared to the corresponding quarter of fiscal year 2023/2024.

\vspace{0.6em}

The value of the Egyptian pound against the US dollar stabilized during the quarter under study (\CurrentQuarterText) at 50.38, representing a recovery of approximately 0.1\% for the pound compared to the previous quarter, while representing a decline of approximately 6\% compared to its value in the corresponding quarter of fiscal year 2023/2024 (Figure 3).
}

% BLOCK:FOOTNOTE
% ---------------------------------------------------------
% FOOTNOTES
% ---------------------------------------------------------
\vfill

\noindent
\rule{5cm}{0.4pt}

\vspace{0.3em}

\noindent
{\fontsize{7}{9}\selectfont
\textsuperscript{17} Central Bank of Egypt, 2025. Press release on the performance of the balance of payments during the period July-March of the fiscal year \CurrentFiscalYear.\\
\textsuperscript{18} Cabinet. 2025. Net foreign assets in the banking sector rose by approximately \$1.2 billion to \$14.7 billion in May 2025, compared to \$13.6 billion in the previous month.
}

\clearpage

% =========================================================
% PAGE 11: PUBLIC FINANCE
% =========================================================

% BLOCK:PAGE_SETUP background=mc_bg.png page=11
% 1. Geometry (Maintains sidebar layout)
\newgeometry{left=5cm, right=1.5cm, top=3cm, bottom=2.5cm}

% 2. Background Graphics
\begin{tikzpicture}[remember picture, overlay]
    % Background Image
    \node[anchor=north west, inner sep=0pt] at (current page.north west) {
        \includegraphics[width=\paperwidth, height=\paperheight]{mc_bg.png}
    };

    % Page Number (Dynamic)
    \node[anchor=south east] at ([xshift=-1cm, yshift=1cm]current page.south east) {
        \fontsize{14}{14}\selectfont \textbf{\thepage}
    };
\end{tikzpicture}

% BLOCK:CHART chart=ch8.png title="Figure 3: Net International Reserves and Exchange Rate"
% ---------------------------------------------------------
% FIGURE 3: RESERVES & EXCHANGE RATE
% ---------------------------------------------------------
\noindent
{\fontsize{10}{12}\selectfont \textbf{\color{ecestitle} Figure 3: Net International Reserves and Exchange Rate}}

\vspace{0.2em}

\noindent
% Using ch8.png to avoid conflict with ch7.png from the previous page
\includegraphics[width=\linewidth]{ch8.png}

\noindent
{\fontsize{7}{9}\selectfont \textit{Sources: Central Bank of Egypt, Monthly Statistical Bulletin, various issues; Ministry of Finance, Monthly Financial Report, various issues.}}

\vspace{1.5em}

% BLOCK:SUBHEADER_LEGEND text="IV. Public Finance"
% ---------------------------------------------------------
% SECTION IV: PUBLIC FINANCE
% ---------------------------------------------------------
\noindent
{\fontsize{11}{14}\selectfont \textbf{IV. Public Finance}}

\vspace{0.5em}

% BLOCK:PARAGRAPH
\noindent
{\fontsize{10}{13}\selectfont
The overall deficit in the State's general budget as a percentage of GDP decreased during the period July-May \CurrentFiscalYear{} to -6.9\% compared to -7.3\% in the corresponding period of the previous fiscal year. This is due to:
}

% BLOCK:BULLET_LIST
\begin{itemize} \itemsep0.6em
    \item \fontsize{10}{13}\selectfont Public revenues rising by 1.3\% to EGP 2,246.6 billion, driven by a 36\% increase in tax revenues.

    \item \fontsize{10}{13}\selectfont Public expenditures increasing by approximately 25.2\% to EGP 3,408.7 billion, while public spending continued to be controlled at a ceiling of EGP 1 trillion (Ministry of Finance 2025).\textsuperscript{22}

    \item \fontsize{10}{13}\selectfont According to the Ministry of Finance, the debt-to-GDP ratio decreased from 89.4\% in June 2024 to 85\% in June 2025, as a result of a decline in domestic debt by approximately eight percentage points and external debt by approximately 2.7 percentage points.
\end{itemize}

% BLOCK:FOOTNOTE
% ---------------------------------------------------------
% FOOTNOTES
% ---------------------------------------------------------
\vfill

\noindent
\rule{5cm}{0.4pt}

\vspace{0.3em}

\noindent
{\fontsize{7}{9}\selectfont
\textsuperscript{22} Ministry of Finance, Monthly Financial Report, July 2025.
}

\clearpage
