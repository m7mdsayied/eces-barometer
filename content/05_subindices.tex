% =========================================================
% PAGE 24: SUB-INDICES PERFORMANCE (Figure 3.1)
% =========================================================

% 1. Geometry
% Standard margins
\newgeometry{left=2cm, right=2cm, top=3cm, bottom=2.5cm}

% 2. Background Image (con_bg.png)
\begin{tikzpicture}[remember picture, overlay]
    
    % A. Main Background
    \node[anchor=north west, inner sep=0pt] at (current page.north west) {
        \includegraphics[width=\paperwidth, height=\paperheight]{con_bg.png}
    };
    
    % B. Page Number (Dynamic)
    \node[anchor=south east] at ([xshift=-1cm, yshift=1cm]current page.south east) {
        \fontsize{14}{14}\selectfont \textbf{\thepage}
    };

\end{tikzpicture}

% ---------------------------------------------------------
% SECTION TITLE
% ---------------------------------------------------------
\noindent
{\fontsize{18}{22}\selectfont \textbf{\color{ecestitle} III. Performance evaluation and expectations \\ according to sub-indices}}

\vspace{1em}

% ---------------------------------------------------------
% SUBSECTION 3.1
% ---------------------------------------------------------
\noindent
{\fontsize{11}{14}\selectfont \textbf{\color{ecestitle} \textit{3.1. Performance evaluation}}}

\vspace{0.8em}

% ---------------------------------------------------------
% BODY TEXT
% ---------------------------------------------------------
\noindent
{\fontsize{10}{13}\selectfont
The indices of production, domestic sales, and capacity utilization exceeded the neutral level during the quarter under study, but saw a decline from the previous quarter, albeit higher than the corresponding quarter. Large firms scored 5 points higher than neutral in the export index, and higher than the previous and corresponding quarters, while recording in commodity inventory lower values than neutral by 7 points, and less than the previous and corresponding quarters.

\vspace{0.8em}

Small and medium enterprises scored higher values than neutral by 5 points in the export index, and less than the previous quarter by 5 points, but better than the corresponding quarter by 8 points. Meanwhile, they scored values below the neutral level by 8 points in the commodity inventory index, and less than the previous quarter by 5 points, although higher than the corresponding quarter by 3 points.
}

\vspace{0.5em}

% ---------------------------------------------------------
% FIGURE 3.1 (Sub-indices Performance)
% ---------------------------------------------------------
\begin{figure}[h!]
    \centering
    % Contains the Production and Sales Indices chart
    \includegraphics[width=\linewidth]{ch20.png}
\end{figure}

% ---------------------------------------------------------
% SOURCE & NOTES
% ---------------------------------------------------------
\vspace{-0.5em}
\noindent
{\fontsize{8}{10}\selectfont
Source: Survey results.\\
* The index for Inventory is inverted to indicate the negative impact of its increase on businesses. Hence, a higher inventory index indicates lower inventory and vice versa.
}

\clearpage

% =========================================================
% PAGE 25: PRICES, WAGES, INVESTMENT, EMPLOYMENT INDICES
% =========================================================

% 1. Geometry (Persist)

% 2. Background Image (con_bg.png)
\begin{tikzpicture}[remember picture, overlay]
    
    % A. Main Background
    \node[anchor=north west, inner sep=0pt] at (current page.north west) {
        \includegraphics[width=\paperwidth, height=\paperheight]{con_bg.png}
    };
    
    % B. Page Number (Dynamic)
    \node[anchor=south east] at ([xshift=-1cm, yshift=1cm]current page.south east) {
        \fontsize{14}{14}\selectfont \textbf{\thepage}
    };

\end{tikzpicture}

% ---------------------------------------------------------
% SECTION 1: PRICES AND WAGES
% ---------------------------------------------------------
\noindent
{\fontsize{10}{13}\selectfont \textbf{The final product price and wage indices for all firms exceeded the neutral level.}}

\vspace{0.5em}

\noindent
{\fontsize{10}{13}\selectfont
The final product prices index for all firms exceeded the neutral level, reflecting price stability during the quarter under study. The wage index exceeded the neutral level by 4 points during the quarter under study, but declined from the previous quarter (Figure 3-2).
}

\vspace{1em}

% ---------------------------------------------------------
% FIGURE 3.2 (Prices/Wages)
% ---------------------------------------------------------
\begin{figure}[h!]
    \centering
    % Price and Production Cost Indices
    \includegraphics[width=\linewidth]{ch21.png}
\end{figure}

\vspace{-0.5em}
\noindent
{\fontsize{8}{10}\selectfont
Source: Survey results.\\
* The index for inputs is inverted to indicate the negative impact of price increases on the overall index. Hence, a lower index indicates higher prices and vice versa.
}

\vspace{1.5em}

% ---------------------------------------------------------
% SECTION 2: INVESTMENT AND EMPLOYMENT
% ---------------------------------------------------------
\noindent
{\fontsize{11}{13}\selectfont \textbf{Stable investment and employment indices for all firms}}

\begin{itemize} \itemsep0.6em
    \item \fontsize{10}{13}\selectfont For large firms, the investment index during the quarter under study scored higher values than the neutral level by one point, and less than the previous and corresponding quarters. The employment index scored values at the neutral level and below the previous and corresponding quartiles.
    
    \item \fontsize{10}{13}\selectfont With regards to small and medium-sized enterprises, the investment index came at the neutral level, recording a decrease compared to the previous quarter. However, it scored the same values as the corresponding quarter. Meanwhile, the employment index scored higher values than neutral by one point, and higher than the previous and corresponding quarters (Figure 3-3).
\end{itemize}

\vspace{0.5em}

% ---------------------------------------------------------
% FIGURE 3.3 (Invest/Employ)
% ---------------------------------------------------------
\begin{figure}[h!]
    \centering
    % Investment and Employment Indices
    \includegraphics[width=\linewidth]{ch22.png}
\end{figure}

\vspace{-0.5em}
\noindent
{\fontsize{8}{10}\selectfont
Source: Survey results.
}

\clearpage

% =========================================================
% PAGE 26: PERFORMANCE EXPECTATIONS (Figure 3.4)
% =========================================================

% 1. Geometry (Persist)

% 2. Background Image (con_bg.png)
\begin{tikzpicture}[remember picture, overlay]
    
    % A. Main Background
    \node[anchor=north west, inner sep=0pt] at (current page.north west) {
        \includegraphics[width=\paperwidth, height=\paperheight]{con_bg.png}
    };
    
    % B. Page Number (Dynamic)
    \node[anchor=south east] at ([xshift=-1cm, yshift=1cm]current page.south east) {
        \fontsize{14}{14}\selectfont \textbf{\thepage}
    };

\end{tikzpicture}

% ---------------------------------------------------------
% SECTION HEADER
% ---------------------------------------------------------
\noindent
{\fontsize{11}{14}\selectfont \textbf{\color{ecestitle} 3.2. Performance Expectations}}

\vspace{0.8em}

% ---------------------------------------------------------
% INTRO TEXT
% ---------------------------------------------------------
\noindent
{\fontsize{10}{13}\selectfont
\textbf{All firms are expected to perform above neutral during the coming quarter, given that most sub-indices are expected to remain stable.}

\vspace{0.8em}

Expectations of large firms regarding production and capacity utilization during the quarter \NextQuarterText{} were higher than neutral, but less than the previous and corresponding quarters. Meanwhile, expectations for domestic sales and exports scored values lower than the previous quarter, but higher than the corresponding quarter. Expectations indicated an increase in the commodity inventory index from the neutral level and both the previous and corresponding quarters.

\vspace{0.8em}

The expectations of small and medium enterprises for domestic sales and inventory came higher than the neutral level and compared to the previous and corresponding quarters, while expectations for production and capacity utilization were lower by one point compared to the previous quarter, although better than the corresponding quarter by 4 points. Meanwhile, expectations for the export index were lower than the previous and corresponding quarters (Figure 3-4).
}

\vspace{0.5em}

% ---------------------------------------------------------
% FIGURE 3.4 (Expectations Size)
% ---------------------------------------------------------
\begin{figure}[h!]
    \centering
    % Production and Sales Indices by Firm Size - Outlook
    \includegraphics[width=\linewidth]{ch23.png}
\end{figure}

% ---------------------------------------------------------
% SOURCE & NOTES
% ---------------------------------------------------------
\vspace{-0.5em}
\noindent
{\fontsize{8}{10}\selectfont
Source: Survey results.\\
* The index for inventory is inverted to indicate the negative impact of its increase on businesses. Hence, a higher inventory index indicates lower inventory and vice versa.
}

\vspace{1.5em}

% ---------------------------------------------------------
% BOTTOM TEXT
% ---------------------------------------------------------
\noindent
{\fontsize{10}{13}\selectfont \textbf{All firms expect final product prices and wages to rise above the neutral level.}}

\vspace{0.8em}

\noindent
{\fontsize{10}{13}\selectfont
The final product price index declined compared to the previous and corresponding quarters, but exceeded the neutral level, driven by the higher input price index from the previous and the corresponding quarters.

\vspace{0.8em}

In the meantime, small and medium enterprises expect a stable wage index at the same values as the previous quarter and higher than neutral level by two points, albeit less than the corresponding quarter.
}

\clearpage

% =========================================================
% PAGE 27: PRICES, WAGES, INVESTMENT OUTLOOK
% =========================================================

% 1. Geometry (Persist)

% 2. Background Image (con_bg.png)
\begin{tikzpicture}[remember picture, overlay]
    
    % A. Main Background
    \node[anchor=north west, inner sep=0pt] at (current page.north west) {
        \includegraphics[width=\paperwidth, height=\paperheight]{con_bg.png}
    };
    
    % B. Page Number (Dynamic)
    \node[anchor=south east] at ([xshift=-1cm, yshift=1cm]current page.south east) {
        \fontsize{14}{14}\selectfont \textbf{\thepage}
    };

\end{tikzpicture}

% ---------------------------------------------------------
% TOP TEXT (Continuation)
% ---------------------------------------------------------
\noindent
{\fontsize{10}{13}\selectfont
by one point. The expectations of large firms came one point higher than the previous quarter and two points lower than the corresponding quarter (Figure 3-5).
}

\vspace{0.5em}

% ---------------------------------------------------------
% FIGURE 3.5 (Expectations Prices)
% ---------------------------------------------------------
\begin{figure}[h!]
    \centering
    % Prices and Production Cost Indices by Firm Size - Outlook
    \includegraphics[width=\linewidth]{ch24.png}
\end{figure}

% ---------------------------------------------------------
% SOURCE & NOTES FOR FIG 3.5
% ---------------------------------------------------------
\vspace{-0.5em}
\noindent
{\fontsize{8}{10}\selectfont
Source: Survey results.\\
* The index for inputs is inverted to indicate the negative impact of price increases on the overall index. Hence, a lower index indicates higher prices and vice versa.
}

\vspace{2em}

% ---------------------------------------------------------
% MIDDLE SECTION: INVESTMENT AND EMPLOYMENT
% ---------------------------------------------------------
\noindent
{\fontsize{10}{13}\selectfont \textbf{The investment and employment expectations indices for all firms remained stable at the neutral level.}}

\vspace{0.8em}

\noindent
{\fontsize{10}{13}\selectfont
All firms expect the investment index to exceed the neutral level as well as the previous and corresponding quarters by one point. They also expect the employment index to exceed the previous quarter by one point, similar to the corresponding quarter, and to record higher values than neutral by one point (Figure 3-6).
}

\vspace{0.5em}

% ---------------------------------------------------------
% FIGURE 3.6 (Expectations Invest/Employ)
% ---------------------------------------------------------
\begin{figure}[h!]
    \centering
    % Investment and Employment Indices by Firm Size - Outlook
    \includegraphics[width=\linewidth]{ch25.png}
\end{figure}

% ---------------------------------------------------------
% SOURCE FOR FIG 3.6
% ---------------------------------------------------------
\vspace{-0.5em}
\noindent
{\fontsize{8}{10}\selectfont
Source: Survey results.
}

\clearpage