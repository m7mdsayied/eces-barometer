% =========================================================
% PAGE 12: BUSINESS BAROMETER INDEX (BBI) - ARABIC
% =========================================================

% 1. Geometry
% Standard margins for the main text area
\newgeometry{left=2cm, right=2cm, top=3cm, bottom=2.5cm}

% 2. Background Image & Graphics
\begin{tikzpicture}[remember picture, overlay]
    
    % A. Main Background (Assuming generic or AR specific background)
    \node[anchor=north west, inner sep=0pt] at (current page.north west) {
        \includegraphics[width=\paperwidth, height=\paperheight]{con_bg_ar.png}
    };
    
    % B. Page Number (Dynamic - Bottom Left for Arabic usually, or keeps Bottom Right)
    % Keeping it Bottom Right to match English style, or move to Bottom Left if desired.
    \node[anchor=south west] at ([xshift=1cm, yshift=1.2cm]current page.south west) {
        \fontsize{14}{14}\selectfont \textbf{\thepage}
    };

    % C. Top Left Gadgets (oev_ar.png)
    % Placed absolutely in the top LEFT corner for Arabic layout
    % Adjusted yshift to -2.5cm to prevent overlap with title
    \node[anchor=north west] at ([xshift=0cm, yshift=-2.5cm]current page.north west) {
        \includegraphics[width=2.5cm]{oev_ar.png}
    };

\end{tikzpicture}

% ---------------------------------------------------------
% TITLE SECTION
% ---------------------------------------------------------
% Minipage width restricts text to the right side, leaving space for the left image
\noindent
\begin{minipage}{0.85\textwidth}
    
    {\fontsize{20}{24}\selectfont \textbf{\color{ecestitle} مؤشر بارومتر الأعمال}}
    
    \vspace{0.5em}
    
    {\fontsize{18}{22}\selectfont \textbf{\color{ecestitle} أولاً: تقييم الأداء والتوقعات وفقاً للمؤشر الإجمالي}}
    
    \vspace{0.8em}
    
    {\fontsize{11}{14}\selectfont \textbf{استمرار ارتفاع مؤشر أداء الأعمال خلال الربع محل الدراسة عن المستوى المحايد، مع تباين الأداء على مستوى القطاعات.}}
    
\end{minipage}

\vspace{1.5em}

% ---------------------------------------------------------
% SUBSECTION 1-1
% ---------------------------------------------------------
\noindent
{\fontsize{12}{15}\selectfont \textbf{\color{ecestitle} \textit{1-1 تطور المؤشر الإجمالي}}}

\vspace{0.5em}

\noindent
{\fontsize{11}{14}\selectfont
تجاوزت قيم مؤشر أداء الأعمال للربع محل الدراسة (\CurrentQuarterTextAr) \textbf{المستوى المحايد} بنقطة واحدة. ويُعزى السبب في ذلك إلى تجاوز معظم المؤشرات للمستوى المحايد لتعكس استقراراً في الأداء الجيد الذي شهده الربع السابق. ومقارنة بالربع السابق، تراجع مؤشر أداء الأعمال بنحو 7 نقاط نتيجة انخفاض حاد في مؤشر الأجور بعد قفزة كبيرة خلال الربع السابق وكذلك تراجع مؤشرات الإنتاج، والمبيعات، ومستوى استغلال الطاقة الإنتاجية، إلا أنها لا تزال أعلى من المستوى المحايد (الشكل 1-1).

\vspace{0.8em}

كما سجل مؤشر توقعات الأداء للربع (\NextQuarterTextAr) فيما أعلى من \textbf{المستوى المحايد} بمقدار 4 نقاط، وأقل من الربع السابق بنقطة واحدة، وأفضل من الربع المناظر بنقطتين، ويُعزى ذلك إلى التوقعات بارتفاع كافة المؤشرات \textbf{عن المستوى المحايد لثباتها} عند الأداء الحالي باستثناء قطاعات السياحة والاتصالات والخدمات المالية التي يتوقع أن تشهد مؤشراتها الفرعية ارتفاعاً خلال الربع القادم (الشكل 1-2).
}

\vspace{1em}

% ---------------------------------------------------------
% CHART (Overall Index)
% ---------------------------------------------------------
\begin{figure}[h!]
    \centering
    % Main Overall Index Chart
    \includegraphics[width=\linewidth]{ch9.png}
\end{figure}

% ---------------------------------------------------------
% FOOTNOTES / SOURCE
% ---------------------------------------------------------
\vspace{0.5em}
\noindent
{\fontsize{8}{10}\selectfont
\textit{المصدر:} نتائج الاستبيان.\\
* البيانات الخاصة بالفترة يناير - مارس 2020 غير متاحة بسبب الإغلاق أثناء الجائحة.\\
** البيانات الخاصة بالفترة أبريل - يونيو 2020 غير متاحة بسبب الإغلاق أثناء الجائحة.
}

\clearpage

% =========================================================
% PAGE 13: CHARTS (Firm Size)
% =========================================================

% 1. Geometry & Page Setup
\newgeometry{left=2cm, right=2cm, top=2cm, bottom=2cm}

% 2. Background Image
\begin{tikzpicture}[remember picture, overlay]
    \node[anchor=north west, inner sep=0pt] at (current page.north west) {
        \includegraphics[width=\paperwidth, height=\paperheight]{con_bg_ar.png}
    };
    \node[anchor=south west] at ([xshift=1cm, yshift=1.2cm]current page.south west) {
        \fontsize{14}{14}\selectfont \textbf{\thepage}
    };
\end{tikzpicture}

% ---------------------------------------------------------
% SECTION 1-2: INDEX BY FIRM SIZE
% ---------------------------------------------------------
\noindent
{\fontsize{12}{14}\selectfont \textbf{\color{ecestitle} 1-2 المؤشر وفقاً لأحجام الشركات}}

\vspace{0.5em}

\noindent
{\fontsize{10}{13}\selectfont
لا يوجد تباين في الأداء بين مختلف أحجام الشركات؛ حيث تجاوز مؤشر أداء الأعمال \textbf{لكافة الشركات المستوى المحايد} خلال الربع محل الدراسة مسجلاً قيماً أقل من الربع السابق، وإن كانت أفضل من الربع المناظر، ويعكس ذلك ثبات الأداء لكافة المؤشرات الفرعية، وعدم تحسنها مع ملاحظة تحسن مؤشر الصادرات للشركات الكبيرة عن الربعين السابق والمناظر (الشكل 1-3).

كما تجاوز مؤشر التوقعات (للربع \NextQuarterTextAr) \textbf{المستوى المحايد لكافة الشركات} مسجلاً قيم أعلى منه بـ 3 نقاط \textbf{للشركات الكبيرة}، ولكنه لا يزال أقل من الربع السابق بـ 4 نقاط، وأفضل من الربع المناظر بنقطة واحدة. أما بالنسبة \textbf{للشركات الصغيرة والمتوسطة} فقد سجل مؤشر التوقعات \textbf{قيماً} أعلى من \textbf{المستوى المحايد} بـ 4 نقاط وبنفس قيم الربع السابق وأفضل من الربع المناظر بنقطتين، ويرجع ذلك إلى التوقعات بثبات الأداء خلال الربع القادم (الشكل 1-4).
}

% ---------------------------------------------------------
% FIGURES 1.3 & 1.4 (Size Charts)
% ---------------------------------------------------------
\begin{figure}[h!]
    \centering
    % Chart: Firm Sizes (Performance & Expectations)
    \includegraphics[width=\linewidth]{ch10.png}
\end{figure}

% ---------------------------------------------------------
% SECTION 1-3: INDEX BY ECONOMIC SECTORS (INTRO)
% ---------------------------------------------------------
\vspace{1em}
\noindent
{\fontsize{12}{14}\selectfont \textbf{\color{ecestitle} 1-3 المؤشر وفقاً للقطاعات الاقتصادية}}

\vspace{0.5em}

\noindent
{\fontsize{10}{13}\selectfont
تجاوز مؤشر الأداء \textbf{لكافة القطاعات} المستوى المحايد \textbf{باستثناء قطاع الصناعات التحويلية} الذي سجل قيماً أقل من المستوى المحايد وأقل أداءً بين جميع القطاعات، بينما جاءت قيم \textbf{قطاع التشييد والبناء} عند المستوى المحايد. وحقق \textbf{قطاع الاتصالات} أفضل أداء خلال الربع محل الدراسة، ومقارنة بالربع السابق، تراجع مؤشر أداء الأعمال \textbf{لكافة القطاعات}، وجاء التراجع الأكثر حدة في \textbf{قطاع السياحة} يليه \textbf{قطاع الصناعات التحويلية} (الشكل 1-5).
}

% ---------------------------------------------------------
% FIGURE 1.5 (Sector Chart - Performance)
% ---------------------------------------------------------
\begin{figure}[h!]
    \centering
    % Chart: Sectors Performance
    \includegraphics[width=\linewidth]{ch11.png}
\end{figure}

\clearpage

% =========================================================
% PAGE 14: DETAILED SECTOR ANALYSIS
% =========================================================

% 1. Geometry
\newgeometry{left=2cm, right=2cm, top=3cm, bottom=2.5cm}

% 2. Background Image
\begin{tikzpicture}[remember picture, overlay]
    \node[anchor=north west, inner sep=0pt] at (current page.north west) {
        \includegraphics[width=\paperwidth, height=\paperheight]{con_bg_ar.png}
    };
    \node[anchor=south west] at ([xshift=1cm, yshift=1.2cm]current page.south west) {
        \fontsize{14}{14}\selectfont \textbf{\thepage}
    };
\end{tikzpicture}

% ---------------------------------------------------------
% INTRO TEXT
% ---------------------------------------------------------
\noindent
{\fontsize{10}{13}\selectfont \textbf{وفيما يلي تحليل لأداء القطاعات الاقتصادية خلال الربع محل الدراسة وفقاً لآراء العينة، ومقارنة بالأداء خلال الربعين السابق والمناظر:}}

\vspace{0.5em}

% ---------------------------------------------------------
% MAIN BULLET POINTS (Past Performance)
% ---------------------------------------------------------
\begin{itemize} \itemsep0.6em
    \item \fontsize{10}{13}\selectfont سجل \textbf{قطاع الصناعات التحويلية أدنى أداء بين جميع القطاعات بقيم} دون \textbf{المستوى المحايد} بنقطتين وأقل من الربع السابق بـ 7 نقاط. وإن كانت أفضل من الربع المناظر بنقطة واحدة. ويرجع ذلك بالأساس إلى انخفاض مؤشرات الإنتاج والمبيعات والصادرات دون المستوى المحايد خلال الربع محل الدراسة، وتراجعها مقارنة بالربع السابق، نظراً لتراجع الطلب خاصة بعد انتهاء الأعياد و انتهاء العام الدراسي وهو ما ظهر بوضوح في صناعات الملابس الجاهزة والصناعات الغذائية.

    \item \fontsize{10}{13}\selectfont سجل \textbf{قطاع التشييد والبناء} قيم \textbf{المستوى المحايد} خلال الربع محل الدراسة وهي قيم أقل من الربع السابق بـ 3 نقاط، وأفضل من الربع المناظر بـ 4 نقاط، ويُعزى ذلك إلى ثبات كافة المؤشرات خلال الربع محل الدراسة، نتيجة استمرار الإقبال على الشراء في المنتجعات والمدن الجديدة، وبدء مشروعات جديدة بالواحات والساحل الشمالي، كما أن لمشروعات الدولة في البنية التحتية والطرق والمدن الجديدة دوراً في انتعاش القطاع بشكل كبير. ولا يزال ارتفاع أسعار المدخلات يمثل أكبر تحديات القطاع خلال الربع الحالي.

    \item \fontsize{10}{13}\selectfont سجل \textbf{قطاع السياحة} قيماً أعلى من \textbf{المستوى المحايد} بـ 7 نقاط وأقل من الربع السابق بـ 9 نقاط، ولكنها أفضل من الربع المناظر بـ 9 نقاط. ويُعزى ذلك إلى انتعاش حركة السياحة الوافدة إلى مصر خلال الربع محل الدراسة، وارتفاع معدلات الإشغال في الفنادق والقرى السياحية، وزيادة الإقبال على حجوزات الطيران، وارتفاع السياحة الخارجية بسبب موسم الحج والعمرة، فضلاً عن زيادة حركة السياحة الداخلية مع قدوم فصل الصيف ونهاية العام الدراسي.

    \item \fontsize{10}{13}\selectfont سجل \textbf{قطاع النقل} قيماً أعلى من \textbf{المستوى المحايد} بـ 6 نقاط بأداء أقل من الربع السابق بـ 5 نقاط وأفضل من الربع المناظر بـ 8 نقاط؛ ويرجع هذا التحسن إلى زيادة حركتي الواردات والصادرات؛ حيث شهدت الفترة محل الدراسة ارتفاع النقل البري المبرد، بالإضافة إلى تزايد النشاط السياحي وخاصة السياحة الشاطئية.

    \item \fontsize{10}{13}\selectfont وسجل \textbf{قطاع الاتصالات} أفضل أداء خلال الربع محل الدراسة محققاً قيماً تجاوزت \textbf{المستوى المحايد} بـ 15 نقطة؛ بأداء أقل من الربع السابق بنقطة واحدة، وأفضل من المناظر بـ 15 نقطة. ويُعزى السبب في ذلك إلى تعافي كافة مؤشرات القطاع، خاصة الصادرات، وتراجع أسعار المدخلات الوسيطة مقارنة بالربع السابق، واستمرار زيادة الطلب على خدمات تكنولوجيا المعلومات، مع استمرار الدولة في التوسع في البنية التحتية، وتطوير مرافق السكة الحديد، وزيادة المناقصات الحكومية في هذا المجال، بالإضافة إلى أن تزايد الطلب على البرامج التعليمية والتدريبية لعب دوراً في انتعاش القطاع.

    \item \fontsize{10}{13}\selectfont سجل \textbf{قطاع الخدمات المالية} قيماً أعلى من \textbf{المستوى المحايد} بـ 12 نقطة بأداء أقل من الربع السابق بـ 5 نقاط، وأفضل من الربع المناظر بـ 26 نقطة. ويرجع ذلك إلى تعافي كافة مؤشرات القطاع نتيجة ارتفاع معدلات التداول بسبب استقرار السياسات النقدية، وتراجع أسعار الفائدة، وزيادة التدفقات المالية بالسوق.
\end{itemize}

\vspace{1em}

% ---------------------------------------------------------
% OUTLOOK SECTION
% ---------------------------------------------------------
\noindent
{\fontsize{10}{13}\selectfont \textbf{وعلى جانب مؤشر التوقعات، فقد تجاوز مؤشر التوقعات لكافة القطاعات المستوى المحايد وحقق مؤشر التوقعات لقطاع الاتصالات أفضل قيم.}}

\begin{itemize} \itemsep0.6em
    \item \fontsize{10}{13}\selectfont \textbf{تجاوز مؤشر التوقعات لقطاع الصناعات التحويلية المستوى المحايد} بـ 4 نقاط بقيم أفضل من الربعين السابق والمناظر. ويرجع ذلك إلى التوقعات بثبات كافة المؤشرات الفرعية خلال الربع القادم.
\end{itemize}

\clearpage

% =========================================================
% PAGE 15: SECTOR OUTLOOK ANALYSIS (Continued)
% =========================================================

% 1. Geometry (Persists)

% 2. Background Image
\begin{tikzpicture}[remember picture, overlay]
    \node[anchor=north west, inner sep=0pt] at (current page.north west) {
        \includegraphics[width=\paperwidth, height=\paperheight]{con_bg_ar.png}
    };
    \node[anchor=south west] at ([xshift=1cm, yshift=1.2cm]current page.south west) {
        \fontsize{14}{14}\selectfont \textbf{\thepage}
    };
\end{tikzpicture}

% ---------------------------------------------------------
% BULLET POINTS (Sector Outlooks)
% ---------------------------------------------------------
\begin{itemize} \itemsep0.6em
    \item \fontsize{10}{13}\selectfont \textbf{تجاوزت التوقعات لقطاع التشييد والبناء المستوى المحايد} بـ 4 نقاط بتوقعات أداء أقل من الربع السابق بـ 7 نقاط، وأفضل من الربع المناظر بنقطتين؛ ويرجع ذلك إلى ثبات كافة المؤشرات واستمرار تنفيذ قانون التصالحات، والثبات النسبي للأسعار سواء المدخلات أو الوحدات، والتوقعات بانخفاض الفائدة في البنوك، ومزيد من الاستثمار في قطاع العقارات، وزيادة المبيعات خلال فصل الصيف بسبب إجازات العاملين في الخارج، وكذلك مشروعات الدولة في البنية التحتية.

    \item \fontsize{10}{13}\selectfont \textbf{جاءت التوقعات لقطاع السياحة} متفائلة بقيم تجاوزت \textbf{المستوى المحايد} بنقطتين، وإن كانت أقل من الربعين السابق والمناظر. ويرجع هذا الارتفاع إلى التوقعات باستمرار نشاط حركة السياحة، خاصة الداخلية، خلال موسم الصيف والإجازات، ونشاط سياحة المؤتمرات. بينما من المتوقع انخفاض السياحة الخارجية بعد انتهاء موسم الحج.

    \item \fontsize{10}{13}\selectfont \textbf{تجاوزت التوقعات لقطاع النقل المستوى المحايد} بنقطتين، وإن جاءت أقل من الربعين السابق والمناظر؛ ويرجع ذلك إلى ثبات كافة المؤشرات، والتوقعات باستقرار حركة التجارة بالإضافة إلى التوقعات باستمرار السياحة على نفس مستوى الربع السابق، كما كان لتحسن باقي القطاعات المختلفة دور مهم في ارتفاع نشاط القطاع.

    \item \fontsize{10}{13}\selectfont وسجل مؤشر التوقعات \textbf{لقطاع الاتصالات} أفضل أداء بين القطاعات؛ حيث تجاوز \textbf{المستوى المحايد} بـ 13 نقطة بقيم أفضل من الربعين السابق والمناظر؛ مدفوعاً بتوقعات بزيادة تصدير الخدمات إلى الأسواق الخارجية، خاصة في أفريقيا، واتجاه الدولة لتطوير البنية التحتية والمرافق من خلال المشروعات القومية، وزيادة الطلب على منتجات تكنولوجيا المعلومات.

    \item \fontsize{10}{13}\selectfont وسجل مؤشر التوقعات \textbf{لقطاع الخدمات المالية} قيماً أعلى من \textbf{المستوى المحايد} بـ 7 نقاط وإن كانت أقل من الربعين السابق والمناظر؛ حيث أظهرت الشركات المدرجة بالبورصة في الفترة السابقة أداءً إيجابياً بالإضافة إلى توقعات بطروحات حكومية جديدة. كما أن تراجع التضخم وارتفاع تدفقات العملة الأجنبية يزيدان من الثقة في السوق المصري (الشكل 1-6).
\end{itemize}

\vspace{0.5em}

% ---------------------------------------------------------
% FIGURE 1.6 (Outlook Chart)
% ---------------------------------------------------------
\begin{figure}[h!]
    \centering
    % Chart: Sectors Expectations
    \includegraphics[width=\linewidth]{ch12.png}
\end{figure}

\vspace{-0.5em}
\noindent
{\fontsize{8}{10}\selectfont المصدر: نتائج الاستبيان.}

\clearpage